\documentclass{ctxdoc}
\usepackage{xcolor}
\usepackage{cdcmd}
\title{\pkg{cdcmd},一跟可扩展的条件化命令宏包}
\author{雾月, Longaster\thanks{Email: \url{longaster@163.com}}}
\date{\today v1.0}


\begin{document}
\maketitle

\pkg{cdcmd} 提供了条件化命令的方式,与 \pkg{styledcmd} 宏包不同,该宏包的某些命令是可 \verb|f-|扩展的,这在某些情况下是有用的。

\section{主要接口}

\begin{function}{\newcondition,\setcondition,\clearcondition}
	\begin{syntax}
		\cs{newcondition} \marg{identifier} \marg{id(s)}
		\cs{setcondition} + \marg{identifier=ids list}
        \cs{clearcondition}\oarg{identifier(s)}
	\end{syntax}
	\cs{newcondition} 定义 \meta{identifier} 的 \meta{ids}。\meta{identifier} 两侧的空格将被移除。

	\cs{setcondition} 设置在当前组中有效的 \meta{id}。不带 \verb|+| 的版本将局部地清除原先设置的 \meta{identifier} 的ids。

    \meta{identifier}、\meta{id} 均不能为 \verb|*|。
    
    \cs{clearcondition} 局部地清除指定的 \meta{identifiers},默认值为 \verb|*|,即清除所有。
\end{function}

\begin{function}{
	\conditionif,\conditioncmd,
	\econditionif,\econditioncmd,
}
	\begin{syntax}
		\cs{conditionif} * \oarg{identifier=ids list} \marg{true} \marg{false}
		\cs{conditioncmd} * \oarg{identifier=ids list} \marg{material}
	\end{syntax}
	当 \meta{identifier=ids list} 的组合使得当前条件为真时,使用 \meta{true}、\meta{material},为假时,使用 \meta{false}。

	不带 \verb|*| 的版本为 \verb|any|,带 \verb|*| 的为 \verb|all|。详见下方的说明。

	\cs{econditionif}、\cs{econditioncmd} 为可 \verb|f|-扩展的。
	\cs{conditionif}、\cs{conditioncmd} 为不可扩展(\tn{protected})的。

	\meta{identifier=ids list} 默认值为 \verb|*|,即在默认情况下将输出 \meta{true}、\meta{material}。
\end{function}

\begin{function}{
	\conditioncase,\conditioncaseTF,
	\econditioncase,\econditioncaseTF
}
	\begin{syntax}
        \cs{conditioncaseTF} * !
        ~~\{
        ~~~~\marg{identifier-ids list case_1} \marg{code_1}
        ~~~~\marg{identifier-ids list case_2} \marg{code_2}
        ~~~~...
        ~~~~\marg{identifier-ids list case_n} \marg{code_n}
        ~~\}
        ~~\marg{true code}
        ~~\marg{false code}
	\end{syntax}
	\verb|!| 未给出时,依次计算 \meta{identifier-ids list cases},直到遇到第一个 \meta{case} 为\emph{真},输出对应的 \meta{code} 与 \meta{true code},若未成功匹配,则输出 \meta{false code}。

	\verb|!| 给出时,依次计算 \meta{identifier-ids list cases},直到遇到第一个 \meta{case} 为\emph{假},输出对应的 \meta{code} 与 \meta{true code},若未成功匹配,则输出 \meta{false code}。
\end{function}

\begin{texnote}
	对 \meta{ids} 的处理使用的是 \LaTeX3 的 \cs{clist_map_...},它们将形如 \verb|{,}| 的list视为空,形如 \verb|{{},}| 视为非空。
\end{texnote}

假定已经使用了如下命令:
\begin{verbatim}
  \newcondition{defined}{}
  \newcondition{paper}{a4,a5,b5}
  \setcondition{paper={a5,b5}}
\end{verbatim}
即 \verb|defined| 已被 \cs{newcondition} 定义且无id,\verb|paper| 已被 \cs{newcondition} 定义且有 \verb|a4,b5,a5| 三个id,其中 \verb|a5,b5| 已被设置,\verb|undefined| 未被定义。

\verb|any| 在满足如下条件之一为真:
\begin{enumerate}
    \item \meta{identifier=ids list} 为 \verb|*|;
    \item \meta{identifier=ids list} 为一个已被定义的 \meta{identifier},且未给出 \meta{id} 值 ,即 \meta{identifier=ids list} 为 \verb|paper| 或 \verb|defined|;
    \item \meta{identifier=ids list} 为一个已被定义的 \meta{identifier},且 \meta{id} 值为 \verb|*|,即 \meta{identifier=ids list} 为 \verb|paper=*| 或 \verb|defined=*|;
    \item \meta{identifier=ids list} 为一个已被定义的 \meta{identifier},且给出的 \meta{ids} 中\emph{有}被设置的id,即 \meta{identifier=ids list} 为 \verb|paper=b5| 或 \verb|paper={a5,b5}| 或 \verb|paper={a5,a0}| (\verb|a0| 未设置,但 \verb|a5| 已设置)。注意 \verb|defined| 给出的任何除 \verb|*| 之外的 \meta{ids} 值都将使得条件为假,即使是空值(\verb|defined=|);
    \item \meta{identifier=ids list} 中\emph{任意一项}满足如上4点之一,如 \verb|paper={a5,a0},undefined|。
\end{enumerate}

\verb|all| 在满足如下条件之一为真:
\begin{enumerate}
    \item \meta{identifier=ids list} 为 \verb|*|;
    \item \meta{identifier=ids list} 为一个已被定义的 \meta{identifier},且未给出 \meta{id} 值 ,即 \meta{identifier=ids list} 为 \verb|paper| 或 \verb|defined|;
    \item \meta{identifier=ids list} 为一个已被定义的 \meta{identifier},且 \meta{id} 值为 \verb|*|,即 \meta{identifier=ids list} 为 \verb|paper=*| 或 \verb|defined=*|;
    \item \meta{identifier=ids list} 为一个已被定义的 \meta{identifier},且给出的 \meta{ids} \emph{均}被设置,即 \meta{identifier=ids list} 为 \verb|paper=b5| 或 \verb|paper={a5,b5}|。注意 \verb|defined| 给出的任何除 \verb|*| 之外的 \meta{id} 值都将使得条件为假,即使是空值(\verb|defined=|);
    \item \meta{identifier=ids list} 中\emph{所有项均}满足如上4点之一,如 \verb|paper={a5,b5},defined|。
\end{enumerate}

\begin{function}{
	\newconditioncommand,
	\renewconditioncommand,
	\provideconditioncommand,
	\declareconditioncommand,
	\neweconditioncommand,
	\reneweconditioncommand,
	\provideeconditioncommand,
	\declareeconditioncommand,
}
	\begin{syntax}
		\cs{newconditioncommand} * \meta{function} \oarg{arg nums} \oarg{default} \marg{code}
		\cs{neweconditioncommand} * \meta{function} \oarg{arg nums} \marg{code}
	\end{syntax}
	这些命令与 \LaTeXe{} 的命令相对应。\cs{newconditioncommand} 命令正如 \LaTeXe{} 的 \tn{newcommand} 命令。它们将定义形如 \verb|\foo+|\marg{identifier=ids list}\meta{args} 的命令。定义的命令的可选参数不可包含 \tn{par}。

	\cs{neweconditioncommand} 命令定义的是可扩展的命令,不可设置默认参数。但可使用 \pkg{xparse} 形式的命令定义可设置默认值的参数。
	
	不带 |*| 的版本定义的命令是 \tn{long} 的。即其参数可以包含多个段落。
\end{function}

\begin{function}{
    \NewConditionCommand ,
    \RenewConditionCommand ,
    \ProvideConditionCommand ,
    \DeclareConditionCommand ,
    \NewExpandableConditionCommand ,
    \RenewExpandableConditionCommand ,
    \ProvideExpandableConditionCommand ,
    \DeclareExpandableConditionCommand ,
}
	\begin{syntax}
		\cs{NewConditionCommand} \meta{function} \marg{arg spec} \marg{code}
	\end{syntax}
	这些命令与 \pkg{xparse} 宏包的命令相对应。它们将定义形如 \verb|\foo+|\marg{identifier=ids list}\meta{args} 的命令。

	\meta{arg spec} 应符合 \pkg{xparse} 宏包相应命令的规则。

	其中带 \verb|+| 的命令为 |all|,不带 \verb|+| 的为 |any|。\meta{identifier=ids list} 的含义如上所述。\meta{args} 为
	由 \meta{arg spec} 定义的参数。
\end{function}


\section{例子}

\iffalse
\fi
{\color{red!90!black}
\begin{verbatim}
  \newcondition{defined}{}
  \newcondition{paper}{a4,a5,b5}
  \setcondition{paper={a5,b5}}
\end{verbatim}
}

{
\newcondition{defined}{}
\newcondition{paper}{a4,a5,b5}
\setcondition{paper={a5,b5}}
\def\cdcmdtest#1->#2 /{\textcolor{red!90!black}{\detokenize{#1}}:~ #1 \newline}
\parindent=0pt\ttfamily
\cdcmdtest\conditionif[*]{t}{f}->t /
\cdcmdtest\conditionif[defined]{t}{f}->t /
\cdcmdtest\conditionif[defined=]{t}{f}->f /
\cdcmdtest\conditionif[defined=*]{t}{f}->t /
\cdcmdtest\conditionif[defined=a]{t}{f}->f /
\cdcmdtest\conditionif[paper={a5,a0},undefined]{t}{f}->t /

\cdcmdtest\conditionif*[*]{t}{f}->t /
\cdcmdtest\conditionif*[defined]{t}{f}->t /
\cdcmdtest\conditionif*[defined={,,}]{t}{f}->f /
\cdcmdtest\conditionif*[defined=*]{t}{f}->t /
\cdcmdtest\conditionif*[defined=a]{t}{f}->f /
\cdcmdtest\conditionif*[paper={a5,a0},undefined]{t}{f}->f /
\cdcmdtest\conditionif*[*,undefined]{t}{f}->f /
\cdcmdtest\conditionif*[paper={a5,b5}]{t}{f}->t /
\cdcmdtest\conditionif*[paper={a5,,b5}]{t}{f}->t /
\cdcmdtest\conditionif*[paper={a5,b6,a5}]{t}{f}->f /
\cdcmdtest\conditionif*[paper={a5,{ },45}]{t}{f}->f /
\cdcmdtest\conditionif*[*,defined,paper={a5,b5}]{t}{f}->t /
}

{\color{red!90!black}
\begin{verbatim}
  \def\truetext{true} \def\falsetext{false}
  \edef\testa{\econditionif[*]{true}{false}}
  \ifx\testa\truetext t\else f\fi
  \ifx\testa\falsetext t\else f\fi
  \strcmp {\econditionif[*]{true}{false}} {true}
  \strcmp {\econditionif[*]{true}{false}} {false}
  \strcmp {\testa} {\truetext}
  \strcmp {\testa} {\falsetext}
\end{verbatim}
}
\def\truetext{true} \def\falsetext{false}
\edef\testa{\econditionif[*]{true}{false}}
\ifx\testa\truetext t\else f\fi
\ifx\testa\falsetext t\else f\fi
\strcmp{\econditionif[*]{true}{false}}{true}
\strcmp{\econditionif[*]{true}{false}}{false}
\strcmp{\testa}{\truetext}
\strcmp{\testa}{\falsetext}

{\color{red!90!black}
\begin{verbatim}
  \conditioncase{
    {paper=a3} {a3}
    {paper=a4} {a4}
    {paper,defined} {pd}
  }
\end{verbatim}
}
\conditioncase{
  {paper=a3} {a3}
  {paper=a4} {a4}
  {paper,defined} {pd}
}

{\color{red!90!black}
\begin{verbatim}
  \conditioncaseTF{
    {paper=a3} {a3}
    {paper=a4} {a4}
    {paper,defined} {pd}
  }{true}{false}
\end{verbatim}
}
\conditioncaseTF{
  {paper=a3} {a3}
  {paper=a4} {a4}
  {paper,defined} {pd}
}{true}{false}

{\color{red!90!black}
\begin{verbatim}
  \conditioncase!{
    {paper=a3} {a3}
    {paper=a4} {a4}
    {paper,defined} {pd}
  }
\end{verbatim}
}
\conditioncase!{
  {paper=a3} {a3}
  {paper=a4} {a4}
  {paper,defined} {pd}
}

{\color{red!90!black}
\begin{verbatim}
  \conditioncaseTF!{
    {paper=a3} {a3}
    {paper=a4} {a4}
    {paper,defined} {pd}
  }{true}{false}
\end{verbatim}
}
\conditioncaseTF!{
  {paper=a3} {a3}
  {paper=a4} {a4}
  {paper,defined} {pd}
}{true}{false}

{\color{red!90!black}
\begin{verbatim}
  \newconditioncommand\longprotectedcdcmd{longprotectedcdcmd}
  \newconditioncommand\longprotectedcdcmdi[1]{longprotectedcdcmdi<#1>}
  \newconditioncommand\longprotectedcdcmdio[1][DFT]{longprotectedcdcmdio<#1>}
  \newconditioncommand*\shortprotectedcdcmd{shortprotectedcdcmd}
  \newconditioncommand*\shortprotectedcdcmdi[1]{shortprotectedcdcmdi<#1>}
  \newconditioncommand*\shortprotectedcdcmdio[1][DFT]{shortprotectedcdcmdio<#1>}

  \setcondition{paper={a4,a5}}
  \longprotectedcdcmd{*}
  \longprotectedcdcmdi{*}{1\par arg}
  \longprotectedcdcmdio{*}
  \longprotectedcdcmdio{*}[1opt]
  \longprotectedcdcmdio{paper=a4}[1opt a4]
  \longprotectedcdcmdio+{paper={a4,a7}}[1opt a4a7]
  \shortprotectedcdcmd{*}
  \shortprotectedcdcmdi{*}{1\par arg}
  \shortprotectedcdcmdio{*}
  \shortprotectedcdcmdio{*}[1opt]
  \shortprotectedcdcmdio{paper=a4}[1opt a4]
  \shortprotectedcdcmdio+{paper={a4,a7}}[1opt a4a7]
\end{verbatim}
}
\newconditioncommand\longprotectedcdcmd{longprotectedcdcmd}
\newconditioncommand\longprotectedcdcmdi[1]{longprotectedcdcmdi<#1>}
\newconditioncommand\longprotectedcdcmdio[1][DFT]{longprotectedcdcmdio<#1>}
\newconditioncommand*\shortprotectedcdcmd{shortprotectedcdcmd}
\newconditioncommand*\shortprotectedcdcmdi[1]{shortprotectedcdcmdi<#1>}
\newconditioncommand*\shortprotectedcdcmdio[1][DFT]{shortprotectedcdcmdio<#1>}

{\ttfamily\setcondition{paper={a4,a5}}\obeylines
\longprotectedcdcmd{*}
\longprotectedcdcmdi{*}{1\par arg}
\longprotectedcdcmdio{*}
\longprotectedcdcmdio{*}[1opt]
\longprotectedcdcmdio{paper=a4}[1opt a4]
\longprotectedcdcmdio+{paper={a4,a7}}[1opt a4a7]
\shortprotectedcdcmd{*}
\shortprotectedcdcmdi{*}{1arg}
\shortprotectedcdcmdio{*}
\shortprotectedcdcmdio{*}[1opt]
\shortprotectedcdcmdio{paper=a4}[1opt a4]
\shortprotectedcdcmdio+{paper={a4,a7}}[1opt a4a7]
}



\section{For package author}

\begin{function}[pTF]{
  \cdcmd_any_if:n,
  \cdcmd_all_if:n,
  \cdcmd_any_if:o,
  \cdcmd_all_if:o,
  \cdcmd_any_if:V,
  \cdcmd_all_if:V,
  \cdcmd_any_if:f,
  \cdcmd_all_if:f,
}
  它们的含义应该是显然的。
\end{function}

\begin{function}[TF]{
  \cdcmd_any_case_true:n,
  \cdcmd_any_case_false:n,
  \cdcmd_all_case_true:n,
  \cdcmd_all_case_false:n,
}
  它们的含义应该是显然的。
\end{function}

\section{实现}

见 \href{./cdcmd.pdf}{cdcmd.pdf}

\end{document}
